% !TEX encoding = UTF-8 Unicode
%Präambel

%Report für große Doukumente. Dieser ist in Kapitel (\chapter{}) aufgeteilt
%\documentclass[12pt, a4paper, ngerman]{report} 

%Article für normale Doumente
\documentclass[12pt, a4paper, ngerman]{article}

%Deutsche Beschreibungen von generiertem Text (table of contents => Inhaltsverzeichnis)
\usepackage[ngerman]{babel}

%Umlaute
\usepackage[utf8]{inputenc}

%Schriftart Helvetica 
\usepackage[scaled]{helvet}

%Seitenränder
\usepackage{geometry}
%top = Abstand nach oben
%left = Abstand nach links
%right = Abstand nach rechts
%bottom= Abstand nach unten
%heapsep= Abstand zwische Kopfzeile und Text
%footskip= Abstand zwischen Text und Fußzeile
\geometry{a4paper, top=25mm, left=30mm, right=25mm, bottom=30mm, headsep=10mm, footskip=12mm}

%Farben nutzen
\usepackage{xcolor}

%Grafiken einbinden
\usepackage{graphicx}

%Zusätzliche Positionsbefehle
\usepackage{float} 

%Die Einrücktiefe bei einem neuen Absatz
\setlength{\parindent}{0pt}

%Fülltext
\usepackage{blindtext}

%Fuer Zitate	
\PassOptionsToPackage{backend=bibtex}{biblatex}
\usepackage[natbib=true,style=numeric]{biblatex}
\usepackage[babel,german=guillemets]{csquotes}
\bibliography{quellen.bib} 

% Aufnahme von \paragraph in das Inhaltsverzeichnis 
\setcounter{tocdepth}{3}  

%Nummerierung vertiefen, \paragraph kommt mit ins Inhaltsverzeichnis
\setcounter{secnumdepth}{4} 

%Feste Tabellen
\usepackage{tabulary}

%caption für nummerierte Tabellenüberschriften
%booktabs für die Steuerung von Linien
\usepackage{caption, booktabs}

%Eigene Kommandos
% Osi Modell
\newcommand{\osi}{ISO/OSI Referenzmodell\xspace}
\newcommand{\fcs}{FCS, Frame Checking Sequence,\xspace}
\newcommand{\punkt}[1]{\begin{itemize} \item #1 \end{itemize}}


\usepackage{bytefield}

%Ende Präambel
	
\begin{document}

\begin{titlepage}
		\begin{center}
			\includegraphics[width=.8\linewidth]{Grafiken/logo_htw.jpg}\\[1cm]    
			\textsc{\LARGE Hochschule für Technik und Wirtschaft \newline Fakultät für Ingenieurwissenschaften}\\[1.5cm]
			\newcommand{\HRule}{\rule{\linewidth}{0.5mm}} \HRule \\[0.4cm] { \huge \bfseries Dokumentation des Ki Praktikums}\\[0.4cm]
			\HRule \\[1.5cm]

			\begin{minipage}{0.4\textwidth}
				\begin{flushleft} \large
					\emph{Projektleiter:} \\
					\emph{Team 1:}Christoph Drost\\
					\emph{Team 2:} Deniz Kadiogullari
					\end{flushleft}
			\end{minipage}
			\hfill
			\begin{minipage}{0.4\textwidth}
				\begin{flushright} \large
					\emph{Betreuer:} \\
					Prof. Dr. Weber \\
				\end{flushright}
			\end{minipage}
			\vfill
			{\large \today}
		\end{center}
	\end{titlepage}


%Inhaltsverzeichnis auf eigener Seite
\tableofcontents
\newpage 

\section{Aufbau des Datensatzes}

Der Datensatz wurde in Form einer CSV Datei zur Verfügung gestellt. Er besteht aus 45 Spalten, die jeweils die Attribute eines Satzes zur beschreiben. \\ \\

\begin{bytefield}[bitwidth=4.1em]{8}
	\bitheader{0-7} \\
	\begin{rightwordgroup}{01}
		\bitbox{1}{00} & \bitbox{1}{01} & \bitbox{1}{02}&\bitbox{1}{03} & \bitbox{1}{04} & \bitbox{1}{05}\bitbox{1}{06} & \bitbox{1}{07}
	\end{rightwordgroup} \\
	\begin{rightwordgroup}{02}
		\bitbox{1}{08} & \bitbox{1}{09} & \bitbox{1}{10}&\bitbox{1}{11} & \bitbox{1}{12} & \bitbox{1}{13}\bitbox{1}{14} & \bitbox{1}{15}
	\end{rightwordgroup} \\
	\begin{rightwordgroup}{03}
		\bitbox{1}{16} & \bitbox{1}{16} & \bitbox{1}{17}&\bitbox{1}{18} & \bitbox{1}{19} & \bitbox{1}{20}\bitbox{1}{21} & \bitbox{1}{21}
	\end{rightwordgroup} \\
	\begin{rightwordgroup}{04}
		 \bitbox{1}{22} & \bitbox{1}{23}&\bitbox{1}{24} & \bitbox{1}{25} & \bitbox{1}{26}\bitbox{1}{27} & \bitbox{1}{28} & \bitbox{1}{29} 
	\end{rightwordgroup} \\
	\begin{rightwordgroup}{05}
		\bitbox{1}{30} & \bitbox{1}{31} & \bitbox{1}{32}&\bitbox{1}{33} & \bitbox{1}{34} & \bitbox{1}{35}\bitbox{1}{36} & \bitbox{1}{37}
	\end{rightwordgroup} \\
	\begin{rightwordgroup}{06}
		\bitbox{1}{38} & \bitbox{1}{39} & \bitbox{1}{40}&\bitbox{1}{41} & \bitbox{1}{42} & \bitbox{1}{43}\bitbox{1}{44} & \bitbox{1}{45}
	\end{rightwordgroup} \\
\end{bytefield}
	
\textbf{Erklärung}
\begin{enumerate}
	\item Loc\_Id, eindeutiger numerischer Wert des Datensatzes.
	\item Gebietstyp. Dadurch wird das Gebiet des Datensatzes beschrieben. Bsp. A3 für ein eigenes Land, L1 für Autobahn.
	\item  Untertyp des Gebiets, dient dazu, das Gebiet genauer zu klassifizieren.
	\item Kurzbezeichnung für Straße, Bsp: A620.
	\item Straßenname, bzw. Gebietsname. Bsp: Messegelände, Ruhrallee
	\item Erster Straßenname. Bsp: Essen-Heisingen
	\item Zweiter Straßennahme ??????
	\item Area-Verweis (über Loc\_Id) auf das Gebiet, in dem die Lokation liegt
	\item 
\end{enumerate}	



\newpage
\listoffigures
\end{document}